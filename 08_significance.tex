\section{Significance of the study}

I have not found anyone else trying to use a code prediction metric in real code reviews
\todo{I am sure someone must have done this. Find it.}
so that might be a complete novel thing. %rto: check this out carefully. 

%rto: language agnostic, but you only use Java?
As discussed in \ref{subsec:comparing_metric}, Linespots is the first metric for bug prediction, that works language-agnostic and at the same time offers a finer granularity than the file level. This is relevant, as most metrics can only be used with object-oriented languages and are mainly studied on Java projects. %rto: Like you plan to do?

Other process metrics have also only been applied to classes or modules in programs and although similar concepts exist in many modern languages the tools have to be written on a per case basis.

I found some papers that work on a `per change' basis, essentially per commit or hunk (part of a commit) which comes closer. See \cite{5431727} for example.

This thesis might also be super interesting as everything I have seen so far has only done bug predictions in Java. (again, someone somewhere probably tried something else, I just have to find it). So this is one of the few (or maybe really the first?) study to apply this to non java things.
% rto: Check my sys rev and other SLRs if you can find any?